% Use Roman numerals (i, ii, iii, etc.) for page numbers in the front matter.
\pagenumbering{roman}

%%%%%%%%%%%%%%%%%%%%%%%%%%%%%%%%%%%%%%%%%%%%%%%%%%%%%%%%%%%%%%%%%
%% TITLE PAGE.
%%%%%%%%%%%%%%%%%%%%%%%%%%%%%%%%%%%%%%%%%%%%%%%%%%%%%%%%%%%%%%%%%
% No headers or footers on the title pag
\thispagestyle{empty}
\begingroup
\noindent
\setstretch{1}
\\
Bachelor's thesis
\\[0.5em]
Degree Programme in Information Technology
\\[0.5em]
\StudyGroup
\\[0.5em]
June 2017
\\[0.5em]
\endgroup 

\begingroup
\flushleft
\setstretch{1.0}
~
\\[5em]
\setlength{\leftskip}{2.25cm}\AuthorName
\\[0.35cm]
\sffamily\fontsize{24}{28.8}\selectfont
\MakeUppercase{\DocumentTitle}
\\[0.35cm]
\sffamily\fontsize{16}{19.2}\selectfont

\normalfont\large

\vfill
\centering
\includegraphics[height=4.93cm]
{Figure-TUAS}
\par
\endgroup

\clearpage

%%%%%%%%%%%%%%%%%%%%%%%%%%%%%%%%%%%%%%%%%%%%%%%%%%%%%%%%%%%%%%%%%
%% COPYRIGHT PAGE.
%%%%%%%%%%%%%%%%%%%%%%%%%%%%%%%%%%%%%%%%%%%%%%%%%%%%%%%%%%%%%%%%%

%\pagestyle{plain}
%\setcounter{page}{2}
%
%\begingroup
%\centering
%\setstretch{1.0}
%\null
%\vfill
%{\sffamily\textcopyright}~2016
%\\[0.5em]
%\AuthorName
%\\[0.5em]
%All Rights Reserved
%\par
%\endgroup
%
%\clearpage

%%%%%%%%%%%%%%%%%%%%%%%%%%%%%%%%%%%%%%%%%%%%%%%%%%%%%%%%%%%%%%%%%
%% DEDICATION PAGE.
%%%%%%%%%%%%%%%%%%%%%%%%%%%%%%%%%%%%%%%%%%%%%%%%%%%%%%%%%%%%%%%%%

%\begingroup
%\centering
%\setstretch{1.0}
%~
%\\[1in]
%\textit{Insert dedication here}
%\par
%\endgroup
%
%\clearpage

%%%%%%%%%%%%%%%%%%%%%%%%%%%%%%%%%%%%%%%%%%%%%%%%%%%%%%%%%%%%%%%%%
%% ACKNOWLEDGMENTS.
%%%%%%%%%%%%%%%%%%%%%%%%%%%%%%%%%%%%%%%%%%%%%%%%%%%%%%%%%%%%%%%%%

%\chapter*{Acknowledgments}
%\addcontentsline{toc}{chapter}{Acknowledgments}
%
%{\color{red}%
%Insert thesis acknowledgments here.
%Thesis acknowledgments typically include research advisers and mentors, thesis committee members, collaborators, and funding sources.}
%
%\lipsum[1-2]
%
%\clearpage

%%%%%%%%%%%%%%%%%%%%%%%%%%%%%%%%%%%%%%%%%%%%%%%%%%%%%%%%%%%%%%%%%
%% ABSTRACT.
%%%%%%%%%%%%%%%%%%%%%%%%%%%%%%%%%%%%%%%%%%%%%%%%%%%%%%%%%%%%%%%%%

\chapter*{}
\addcontentsline{toc}{chapter}{Abstract}
\begingroup
\noindent
\setstretch{1}
\\
Bachelor's thesis
\\[0.5em]
Degree Programme in Information Technology
\\[0.5em]
\StudyGroup
\\[0.5em]
2017
\\[1em]
\AuthorName
\\[0.35cm]
\sffamily\fontsize{14}{12}\selectfont
\MakeUppercase{\DocumentTitle}
\\[0.2em]
\endgroup

The purpose of this thesis was to analyze the support and functionality of the Internet Protocol, version 6, on cloud computing environments, as well as demonstrating a demo application showing the possible advantages of using only IPv6.

First the thesis begins presenting several cloud computing vendors, and showing what is the current support offered in their network stacks, both private and public. After that, the thesis covers how a private cloud can be implemented using IPv6, both for the tenant network and the administration networks. 
There are several limitations on some of the providers, and the software environment is immature; this thesis then examines the necessary changes required to offer native IPv6 networking, when possible.

Finally, this thesis  presents a theoretical model of a networked application that can take advantage of the move to the IPv6 network stack. 

The version 6 of the Internet Protocol is a required change, and the whole Internet is required to adapt; the thesis' topics shows how to create services on the new generation of the Internet, and it's advantages and problems.

\clearpage

%%%%%%%%%%%%%%%%%%%%%%%%%%%%%%%%%%%%%%%%%%%%%%%%%%%%%%%%%%%%%%%%%
%% TABLE OF CONTENTS (TOC), LISTS OF FIGURES, TABLES, ETC.
%%%%%%%%%%%%%%%%%%%%%%%%%%%%%%%%%%%%%%%%%%%%%%%%%%%%%%%%%%%%%%%%%

\tableofcontents

\listoffigures

\listoftables

\clearpage

% Use Arabic numerals (1, 2, 3, etc.) for subsequent page numbers.
\pagenumbering{arabic}
